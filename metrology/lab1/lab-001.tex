\documentclass[12pt, a4paper]{article}
\usepackage[a4paper, includeheadfoot, mag=1000, left=2cm, right=1.5cm, top=1.5cm, bottom=1.5cm, headsep=0.8cm, footskip=0.8cm]{geometry}
\usepackage[english, russian]{babel}
\usepackage{indentfirst}
\usepackage{graphicx}
\usepackage{float}
\usepackage{fancyhdr}
\usepackage{amsmath}
\usepackage{listings}
\usepackage{xcolor}
\usepackage{amsfonts} % Добавлен пакет для математических символов

% Стиль разделов
\usepackage{titlesec}
\titleformat{\section}{\large\bfseries}{\thesection}{1em}{}
\titleformat{\subsection}{\bfseries}{\thesubsection}{1em}{}

% Настройка списков
\usepackage{enumitem}
\setlist{nosep}

\begin{document}
	
	% Титульная страница
	\begin{titlepage}
		\newgeometry{left=2cm, right=2cm, top=2cm, bottom=2cm}
		\begin{center}
			
			% Логотип (раскомментируйте и укажите путь к вашему файлу)
			%\includegraphics[width=0.3\textwidth]{itmo_logo.jpg}\\[1cm]
			
			{\LARGE\textbf{Национальный исследовательский университет ИТМО}}\\[0.5cm]
			{\large\textbf{Факультет программной инженерии и компьютерной техники}}\\[1.5cm]
			
			\vfill
			
			{\Large\textbf{Лабораторная работа №1}}\\[0.5cm]
			{\Large по дисциплине}\\[0.3cm]
			{\Large\textbf{Метрология, стандартизация и сертификация}}\\[0.5cm]
			{\large\text{Вариант: 12}}\\[0.5cm]
			
			\vfill
			
			\begin{flushright}
				\large
				\begin{tabular}{rl}
					\textbf{Выполнил:} & Терновский И. Е. \\
					\textbf{Группа:} & P3432 \\
					\textbf{Преподаватель:} & Рассадина А. А. \\
				\end{tabular}
			\end{flushright}
			
			\vfill
			
			{\large г. Санкт-Петербург}\\[0.3cm]
			{\large 2025 г.}
			
		\end{center}
		\restoregeometry
	\end{titlepage}
	
	
	\section*{Задание}
	
	Записать оценку измеряемой величины с учетом случайной и
	систематической погрешностей, если производились прямые измерения.
	
	Вариант 12:
	
	Внешнего диаметра крутильного маятника с помощью линейки с ценой деления
	1 мм
	
	\section*{Измерения}
	
	\begin{center}
		\begin{tabular}{|c|c|}
			\hline
			N & Значение, мм \\\hline
			1 & 25.1 \\\hline
			2 & 24.9 \\\hline
			3 & 25.0 \\\hline
			4 & 24.9 \\\hline
			5 & 24.9 \\\hline
		\end{tabular}
	\end{center}
	
	\section*{Ход работы}
	
	\subsection*{Устранение или учет известных систематических погрешностей}
	
	О системных погрешностях ничего не известно, поэтому переходим к пункту 2.
	
	\subsection*{Вычисление среднего значения}
	
	За эту оценку принимают среднее арифметическое значение по формуле:
	
	$$\bar{x} = \frac{1}{n} \sum_{i = 1}^{n} x_i$$
	
	$\bar{x} = \frac{1}{5} \cdot (25.1 + 24.9 + 25.0 + 24.9 + 24.9) = 24.96$ мм.
	
	\subsection*{Вычисление среднего квадратического отклонения}
	
	$$S_x = \sqrt{\frac{1}{n-1} \sum_{i = 1}^{n} (x_i - \bar{x})^2}$$
	
	Вычисляем сумму квадратов отклонений:
	\begin{align*}
		(25.1 - 24.96)^2 &= 0.14^2 = 0.0196 \\
		(24.9 - 24.96)^2 &= (-0.06)^2 = 0.0036 \\
		(25.0 - 24.96)^2 &= 0.04^2 = 0.0016 \\
		(24.9 - 24.96)^2 &= (-0.06)^2 = 0.0036 \\
		(24.9 - 24.96)^2 &= (-0.06)^2 = 0.0036 \\
		\sum &= 0.0320
	\end{align*}
	
	$S_x = \sqrt{\frac{1}{4} \cdot 0.0320} = \sqrt{0.008} = 0.0894$ мм.
	
	\subsection*{Среднеквадратическое отклонение среднего арифметического}
	
	$$S_{\bar{x}} = \frac{S_x}{\sqrt{n}}$$
	
	$S_{\bar{x}} = \frac{0.0894}{\sqrt{5}} = \frac{0.0894}{2.2361} = 0.0400$ мм.
	
	\subsection*{Исключение грубых погрешностей}
	
	$$G_1 = \frac{|x_{\text{max}} - \bar{x}|}{S_x}; \quad G_2 = \frac{|\bar{x} - x_{\text{min}}|}{S_x}$$
	
	$G_1 = \frac{|25.1 - 24.96|}{0.0894} = \frac{0.14}{0.0894} = 1.566$
	
	$G_2 = \frac{|24.96 - 24.9|}{0.0894} = \frac{0.06}{0.0894} = 0.671$
	
	$G_T = 1.715$ для $q = 5\%$ и пяти измерений.
	
	$G_1 \leq G_T$, поэтому $x_{\text{max}}$ не считаем промахом.
	
	$G_2 \leq G_T$, поэтому $x_{\text{min}}$ не считаем промахом.
	
	\subsection*{Доверительные границы случайной погрешности}
	
	$$\epsilon = t \cdot S_{\bar{x}}, \quad t [P = 95\%; n = 5] = 2.776$$
	
	$\epsilon = 2.776 \cdot 0.0400 = 0.111$ мм.
	
	\subsection*{Учет систематической погрешности}
	
	Для линейки с ценой деления 1 мм систематическая погрешность обычно принимается равной половине цены деления: $\theta = 0.5$ мм.
	
	\subsection*{Учет полной абсолютной погрешности прямого измерения}
	
	\subsubsection*{Абсолютная погрешность}
	
	$$\Delta \bar{x} = \sqrt{\epsilon^2 + \theta^2}$$
	
	$\Delta \bar{x} = \sqrt{0.111^2 + 0.5^2} = \sqrt{0.012321 + 0.25} = \sqrt{0.262321} = 0.512$ мм.
	
	\subsubsection*{Относительная погрешность}
	
	$$\delta x = \frac{\Delta \bar{x}}{\bar{x}} \cdot 100\%$$
	
	$\delta x = \frac{0.512}{24.96} \cdot 100\% = 2.05\%$
	
	\subsection*{Запись результата}
	
	Округляем до одной значащей цифры, так как первая цифра погрешности > 3
	$$x = 25.0 \pm 0.5 \text{ мм}$$
	
	\section*{Вывод}
	
	В результате выполнения данной лабораторной работы была произведена
	запись оценки измеряемой величины с учетом случайной и систематической
	погрешностей по результатам прямых измерений. Полученный результат
	$24.96 \pm 0.51$ мм с относительной погрешностью 2.05\% соответствует
	требованиям точности для измерений линейкой с ценой деления 1 мм.
	
\end{document}